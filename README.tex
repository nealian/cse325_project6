\documentclass[paper=a4, fontsize=11pt]{scrartcl}
\usepackage[T1]{fontenc}
\usepackage{fourier} % We're using the Utopia font because it's great.
\usepackage{enumerate} % Allows for custom enumeration types
\usepackage{tikz} % Graphics powered by TikZ!
\usepackage{xcolor} % More colors
\usepackage{listings} % For shell output
\usepackage{nameref} % Reference sections by name, since we're avoiding section numbering

\usetikzlibrary{positioning} % Relative positioning of nodes
\usetikzlibrary{automata} % Handy stuff for state machine diagrams
\usetikzlibrary{shapes.multipart} % Multiline node labels

\tikzstyle{every state}=[fill=orange!30, draw=orange, very thick]
\tikzstyle{free}=[fill=white]
\tikzstyle{link}=[->, draw=cyan, very thick]

\lstdefinestyle{ShellStyle} {
  basicstyle=\small\ttfamily,
  numbers=none,
  frame=tblr,
  columns=fullflexible,
  backgroundcolor=\color{blue!10},
  linewidth=0.9\linewidth,
  xleftmargin=0.1\linewidth
}

\newcommand{\horrule}[1]{\rule{\linewidth}{#1}} % Horizontal rule with weight arg

\title{
  \normalfont \normalsize 
  \textsc{New Mexico Tech} \\ [25pt]
  \horrule{0.5pt} \\[0.4cm]
  \huge CSE 325 --- Lab Project 6 \\ Simple File System \\
  \horrule{2pt} \\[0.5cm]
}

\author{Rob Kelly \& Ian Neal \\ SANIC TEEM}
\date{\normalsize\today}

\begin{document}
\maketitle

%%% INTRODUCTION %%%
% TODO
% The SANIC TEEM Maniac Middle Management Memory Minder (hereafter referred to as \textit{the project}) is a highly-configurable simulated memory manager with a number of different contiguous memory allocation strategies available for use. Specifically, these allocation strategies are as follows:

% \begin{description}
%   \item[First Fit], which will allocate the suitable\footnote{Here, a ``suitable'' block is a block that is not currently allocated and is large enough to fit the new data.} block that is sequentially first in memory,

%   \item[Best Fit], which will allocate the smallest suitable block,

%   \item[Worst Fit], which will allocate the largest suitable block, and

%   \item[Next Fit], which will allocate the suitable block that is sequentially first in memory \textbf{after the last block allocated,} wrapping around to the start of addressable space upon reaching the end.
% \end{description}

% Most of the codebase of this project was given as part of the lab assignment. We have implemented the functionality of the project in the following files:

% \begin{itemize}
%   \item Implementation of all described functionality in \texttt{mymem.c}.

%   \item This README, the \LaTeX\hspace{0em} source of which may be found in \texttt{README.tex}.

%   \item Build targets for the added files and this README in the \texttt{Makefile}.
% \end{itemize}

%%% BUILDING %%%
\section*{Building}
\begin{itemize}
  \item \texttt{make} to build the project \& testing harness normally.

  \item \texttt{make doc} to build this README. Requires \texttt{pdflatex} and a number of \LaTeX\hspace{0em} packages, all of which are included in the popular \textbf{TeX Live} distribution.

  \item \texttt{make clean} to clean up temporary files, build files, and output.
\end{itemize} 

%%% USAGE %%%
\section*{Usage}
\texttt{./fs\_test}

%%% DESIGN %%%
\section*{Design \& Implementation}
% TODO

%%% BUGS %%%
\section*{Known Bugs}
% TODO

\end{document}
